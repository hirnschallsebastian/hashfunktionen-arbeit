\documentclass[12pt,a4paper]{scrartcl}
\usepackage[utf8]{inputenc}
\usepackage{amsmath}
\usepackage{amsfonts}
\usepackage{amssymb}
\usepackage{graphicx}
\usepackage{color}
\usepackage{multirow}
\usepackage{longtable}
\usepackage[table,xcdraw,dvipsnames]{xcolor}
\usepackage[utf8]{inputenc}
\usepackage[ngerman]{babel}

\begin{document}\noindent
	{\huge \bfseries Begleitprotokoll}
	\newline
	\par\noindent
	\textbf{Name des Schülers:} Sebastian Hirnschall\\
	\textbf{Thema der Arbeit:}\par Funktionsweise und Schwachstellen von kryptographischen Hashfunktionen\\
	\textbf{Name der Betreuungsperson:} Mag. Christian Filipp\\
	
	
	\begin{longtable}{| p{.15\textwidth} | p{.85\textwidth} |}
		\hline
		\rowcolor[HTML]{C0C0C0} 
		\multicolumn{1}{|c|}{\cellcolor[HTML]{C0C0C0}Datum} & \multicolumn{1}{c|}{\cellcolor[HTML]{C0C0C0}\begin{tabular}[c]{@{}c@{}}Vorgangsweise, ausgeführte Arbeiten,\\ verwendete Hilfsmittel, aufgesuchte Bibliotheken,\dots\end{tabular}} \\ \hline
		\rowcolor[HTML]{EFEFEF} 
		01.12.2016 & Empirischer Teil der Arbeit fertiggestellt \\ \hline
		\rowcolor[HTML]{EFEFEF} 
		05.12.2016 & \begin{tabular}[c]{@{}l@{}}Abschnitt 2.1 Kryptographische Hashfunktionen\\ -Definition\\ -Beweis\\ -Literatur: Stinson und Schneier\\ -Bibliothek: TU Wien\end{tabular} \\ \hline
		\rowcolor[HTML]{EFEFEF} 
		06.12.2016 & Änderung an Abschnitt 2.1 \\ \hline
		\rowcolor[HTML]{EFEFEF} 
		08.12.2016 & \begin{tabular}[c]{@{}l@{}}Abschnitt 2.1 Kryptographische Hashfunktionen\\ -Theorem\\ -Beweis\\ -Abschnitt 2.1 fertiggestellt\end{tabular} \\ \hline
		\rowcolor[HTML]{EFEFEF} 
		10.12.2016 & \begin{tabular}[c]{@{}l@{}}-Bruteforce Code optimiert\\ -Änderungen an Abschnitt 2.1\end{tabular} \\ \hline
		\rowcolor[HTML]{EFEFEF} 
		12.12.2016 & -Literaturverzeichnis mit Biblatex \\ \hline
		\rowcolor[HTML]{EFEFEF} 
		15.01.2017 & \begin{tabular}[c]{@{}l@{}}Abschnitt 3.1 MD4\\ -Beschreibung des MD4 Algorithmus\end{tabular} \\ \hline
		\rowcolor[HTML]{EFEFEF} 
		16.01.2017 & \begin{tabular}[c]{@{}l@{}}Abschnitt 3.1 MD4\\ -Angriffe\\ -Schwachstellen\end{tabular} \\ \hline
		\rowcolor[HTML]{EFEFEF} 
		17.01.2017 & \begin{tabular}[c]{@{}l@{}}Abschnitt 3.2 MD5\\ -MD5 Schritte\\ -MD5 Änderungen zu MD4 (Rivest-rfc1320)\end{tabular} \\ \hline
		\rowcolor[HTML]{EFEFEF} 
		18.01.2017 & \begin{tabular}[c]{@{}l@{}}Abschnitt 3.1 MD4\\ -Beispielrechnung - händisch\end{tabular} \\ \hline
		\rowcolor[HTML]{EFEFEF} 
		22.01.2017 & \begin{tabular}[c]{@{}l@{}}Eine frühe Fassung\\ -SHA\\ -Bitoperatoren\\ -Abschnitt 2 geändert\\ -Unterschiede SHA MD5\\ -Markow-Kette Änderung\\ -Bruteforce Code erklärt\\ -Anhang\\ -Kapitel Verwendung gestrichen um 60Tsd. \\ Zeichen nicht zu überschreiten\end{tabular} \\ \hline
	\end{longtable}
	
	\begin{longtable}{| p{.15\textwidth} | p{.85\textwidth} |}
		\hline
		\rowcolor[HTML]{C0C0C0} 
		\multicolumn{1}{|c|}{\cellcolor[HTML]{C0C0C0}Datum} & \multicolumn{1}{c|}{\cellcolor[HTML]{C0C0C0}\begin{tabular}[c]{@{}c@{}}Besprechungen mit der betreuenden Lehrperson, \\ Fortschritte, offene Fragen, Probleme, nächste Schritte\end{tabular}} \\ \hline
		\rowcolor[HTML]{EFEFEF} 
		22.06.2016 & \begin{tabular}[c]{@{}l@{}}über Sommer:\\ -Theorieteil (Bücher aus TU und Vorträge)\\ -praktischer Teil über Vorträge\\ \\ Aufbau:\\ 1. Hashfunktionen (was? + Programmcode, Funktionen, \\ Schwachstellen)\\ 2. Angriffsmethoden (Vergleich, Muster der PW)\\ Analye bereits im Laufen\\ \\ ca.50:50 (Theorie - Praxis)\end{tabular} \\ \hline
		\rowcolor[HTML]{EFEFEF} 
		29.09.2016 & \begin{tabular}[c]{@{}l@{}}-Vorstellen der LaTex-Vorlage (selbst erstellt) – ist O.K.\\ -Besprechen der Zitierweise: direkte Zitate (eingerückt und kursiv)\\ =\textgreater genaues Zitieren in Literaturverzeichnis\\ -Indirekte Zitate: mit Zusatz („Vergleiche“)\\ -selbst erstellte Abbildungen + Code mit Hinweis darauf\\ -In Kopfzeile reicht Hauptkapitel\\ -Formulierung mit „man“ und „ich“ möglichst vermeiden\\ (ausgenommen mathematische Erklärungen)\\ -Zeichenzählen von PDF zu normalem Text\\ (da sonst Sourcecode nicht  mitgezählt werden würde)\end{tabular} \\ \hline
		\rowcolor[HTML]{EFEFEF} 
		12.01.2017 & \begin{tabular}[c]{@{}l@{}}-Definitionen, Beweis und Protokoll (5 Schritte) direkt aus Buch\\ übernommen (Hinweis darauf in Fußnote)\\ -bereits besprochen: kryptographischen Hashfunktionen und\\ praktischer Teil (Hash entschlüsseln + Theorie zu Markow-Ketten)\\ -noch zu erledigen:\ Funktionsweise von Hash-Funktionen und\\Hashfunktionen im Vergleich\\ -bis 31.1.: Endfassung -\textgreater Rückmeldung bis 3.2.\\ -letzter Abgabetermin: 17.2. (in 3 fach gebundener Ausfertigung)\\ -Termin zur Besprechung der Präsentation: 2.3. / 13: 20 (Columbus)\end{tabular} \\ \hline
	\end{longtable}\noindent
	Die Arbeit hat eine Länge von $57~890$ Zeichen.
	
	\vspace{100pt}
	\noindent\rule{5cm}{.4pt}\hfill\rule{5cm}{.4pt}\par
	\noindent 
	\begin{minipage}{5cm}
		\centering
		Datum, Ort
	\end{minipage}  
	\hfill 
	\begin{minipage}{5cm}
		\centering
		Sebastian Hirnschall
	\end{minipage}
	
	\newpage
\end{document}